\documentclass{project}

\usepackage[colorinlistoftodos]{todonotes}
\usepackage[pdfauthor={Group 14},pdftitle={Software Engineering Group Project, Design Specification},pdftex]{hyperref}

\begin{document}

\title{Software Engineering Group Project}
\subtitle{Design Specification}
\author{Group 14}     
\shorttitle{Design Specification}
\version{0.2}
\status{Draft}
\date{2013-12-05}
\configref{SE-14-DESIGN}

\maketitle

\tableofcontents

\newpage

\section{INTRODUCTION}
\subsection{Purpose of this Document}
The purpose of this document is to describe the design of the Walking Tour
Creator and the Walking Tour Displayer applications. It should be read in the
context of the Group Project, taking into account the details of the group
project assignment and the group project quality assurance (QA) plan.
\cite{se.qa.05a} 


\subsection{Scope}
The design specification gives a detailed explanation of how the individual
components of the Walking Tour Creator and the Walking Tour Displayer
applications should be implemented.

This document should be read by the Android application development team and the
website development team.


\subsection{Objectives}

The main objective of this document are:

\begin{itemize}
\item To describe the main components of the Walking Tour Creator and the
Walking Tour Displayer
\item To show the dependencies between the components and the communication
between the application and web server
\item To provide interface details for each of the main classes in the Walking
Tour Creator
\end{itemize}

\newpage

\section{DECOMPOSITION DESCRIPTION}
\subsection{Programs in system}
The system is composed of two programs:
\begin{itemize}
\item The Walking Tour Creator Android application
\item The Walking Tour Displayer website
\end{itemize}


\subsubsection{Walking Tour Creator}
The Walking Tour Creator is an application to allow the user can create walks on
a mobile device for them to be uploaded to the server. It implements the
requirements (FR1), (FR2), (FR3), (FR4), (FR5), (FR6), (FR7), (FR9), (PR1) and
must conform with the requirements (EIR1), (PR2), (DC1), (DC2).\cite{se.qa.rs}

This application will use a GUI as a means for the user to create a walk. The
user will be able to walk around an area and the GPS component of their device
will capture the route that they take. Along the route, they can mark points of
interest, which contains a description and optionally a photo. Once the user has
completed the walk, they can choose to either discard their creation or upload
it to the Walking Tour Displayer.

\subsubsection{Walking Tour Displayer}
The Walking Tour Displayer is website for viewing walks created by users via the
Walking Tour Creator. It implements the requirements (FR8), (FR9), (PR1) and
must conform to the requirements (EIR1), (PR2), (DC1), (DC2), (DC3).
\cite{se.qa.rs}

The walks are stored in a MySQL database with the attributes specified in (DC3).
The database will be queried by the website and allow the user to select a walk.
The selected walk would then be displayed on a map, and the user can select a
points of interest along the route of the walk.

\newpage

\subsection{Significant classes in each program}
\subsubsection{Significant classes in Walking Tour Creator}

\emph{MainAppActivity}. This is the main class of the application.

\emph{WalkCreatorActivity}. This is an activity panel holding the UI for
recording the walk.

\emph{Walk}. This contains a list of the coordinates of the route that the user
is creating, and the metadata of the walk.

\emph{PointOfInterest}. This contains a point of interest in the route, which is
a coordinate with a description and optionally a photo of what is in that area.

\emph{LocationLoggerService}. This is a service that runs in the background
recording the user's walk.


\subsubsection{Significant functions in Walking Tour Displayer}
\todo[inline, color=red!40]{Add the functions to be used on the website}


\subsection{Mapping requirements onto classes}
\begin{tabular}{|l |l |}
\hline
\emph{Requirement} & \emph{Classes providing requirement} \\
\hline
FR1 & MainAppActivity \\
\hline
FR2 & WalkCreatorActivity, Walk \\
\hline
FR3 & WalkCreatorActivity, Walk, PointOfInterest, LocationLoggerService \\
\hline
FR4 & WalkCreatorActivity \\
\hline
FR5 & WalkCreatorActivity \\
\hline
FR6 & WalkCreatorActivity \\
\hline
FR7 & MainAppActivity \\
\hline
FR8 & N/A \\
\hline
FR9 & WalkCreatorActivity \\
\hline
\end{tabular}


\newpage


\section{DEPENDENCY DESCRIPTION}
\subsection{Component Diagrams}

\subsubsection{Component Diagram for Walking Tour Creator}
\todo[inline, color=red!40]{Add image from wiki}

\subsubsection{Component Diagram for Walking Tour Displayer}
\todo[inline, color=red!40]{Add image from wiki}

\subsection{Inheritance Relationships}
The following class dependencies exist in the Walking Tour Creator:
\begin{itemize}
\item TODO
\item TODO
\end{itemize}

\todo[inline, color=red!40]{State what classes depend on what in the list above}

\newpage

\section{INTERFACE DESCRIPTION}
\todo[inline, color=red!40]{Copy the interfaces from the wiki into the sections below}
\subsection{Foo interface specification}
\subsection{Foo interface specification}
\subsection{Foo interface specification}

\newpage

\section{DETAILED DESIGN}
\subsection{Sequence diagrams}
\todo[inline, color=red!40]{Add image from wiki}

\newpage

\subsection{Significant algorithms}
\subsubsection{LocationLoggerService algorithm}

This encompasses several aglroithms. First it decides wich method it picks by
looking at averages over several points and looking at the standard deviation to
see if any anomolies might indicate a change in direction. This is separated
into several different algorithms:

\textbf{Straight line} \\
No change of direction across several points (likely 4 or more, but this could
be changed in the algorithms implementation as a variable) will cause the points
in the middle to be removed. This will look at the standard deviation of points
and general direciton of points to ensure it's not slowly going round a corner.

\textbf{Corner} \\
It will be possible to test for corners above 60� easily by testing for a
straight line followed by another straight line in a completely different
direction. This will probably rely on the previous algorithm to thin out some
points before this can take effect.

\textbf{Slight turn} \\
Detecting a slight turn with a low angle of difference will be difficult. It
will be quite similar to detecting a corner but will likely require more points
of data to detect that the change of direction is not just an anomaly in a
straight line. We can make a vector out of the direction of each 3 points in
succession with a direction and detect when several lines in one direction are
suddenly followed by several lines in a slightly different direction.

\textbf{Rounded turn} \\
Detecting a rounded turn will be the most challenging algorithm. This will be
used for large, sweeping corners and will most likely require either a large
loop or some recursion. Due to the constant change of direction across the turn
this will not be picked up by any of the other algorithms. This itself will
ignore the corner and leave the points in place, allowing the corner's GPS
points to be preserved.

\subsection{Significant data structures}
The data structures we will be using are the IPointOfInterest interface and the
IWalk interface.

The IWalk interface will be implemented by a class and will hold all of the
points of interest. It is the current plan to use a LinkedList due to it being
dynamic in size. If the team member implementing the interface has a compelling
enough reason to use an alternative means of storing them then they can do. Due
to it being an interface this will be up to whoever implements it - the public
methods will remain the same.

The IPointOfInterest will be used to store the data about a point of interest.
For storing the pictures it is again likely that a LinkedList will be used.

The Location objects will be stored in a LinkedList in the AGPSLocationLogger
whilst they are being recorded. This will allow us to quickly and easily add new
Location objects to the end of it. Unlike an array, we will not need to
repeatedly recreate the list with a larger amount of memory each time a new
location is added.

\clearpage
\addcontentsline{toc}{section}{REFERENCES}
\begin{thebibliography}{5}
\bibitem{se.qa.rs} \emph{Software Engineering Group Projects}
Requirements Specification. \\
C. J. Price and B.P.Tiddeman, SE.QA.RS. 1.4 Release.
\bibitem{se.qa.05a} \emph{Software Engineering Group Projects}
Design Specification Standards. \\
C. J. Price, N.W.Hardy, B.P.Tiddeman, SE.QA.05a. 1.7 Release.
\end{thebibliography}
\addcontentsline{toc}{section}{DOCUMENT HISTORY}
\section*{DOCUMENT HISTORY}
\begin{tabular}{|l | l | l | l | l |}
\hline
Version & CCF No. & Date & Changes made to Document & Changed by \\
\hline
0.1 & N/A & 2013-12-04 & Initial creation & tht5 \\
\hline
0.2 & N/A & 2013-12-04 & Moved document to project template & jmt14 \\
\hline
\end{tabular}
\label{thelastpage}

\end{document}