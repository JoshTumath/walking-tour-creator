\documentclass{project}

%---------- Edit for current document 
\usepackage[pdfauthor={Lars H Lunde},pdftitle={Test Specification Formal Review},pdftex]{hyperref}
\usepackage{listings}
\begin{document}
\lstset{language=tex}
\title{JSON Layout}

%---------- Edit for current document 
\subtitle{QA Document}
\author{Group 14}     
\shorttitle{JSON}
\version{1.0}
\status{Final}
\date{\today}
\configref{SE-14-JSON-QA}

\maketitle
\tableofcontents
\newpage

%------------------- Document text
%-------- This section and its 3 subsections are mandatory for ANY document

\section{INTRODUCTION}

\subsection{Purpose of this Document}
The purpose of this document is to create a useful template for JSON documents.
This is required as they are required to match format between web and app to operate correctly.

\subsection{Scope}
This document will cover the possible fields and values that will be created by the WTC application
and sent to the website. The developers of the JSON parts of the application and website are the
people who should read this.

Documents relevenat to this are SE-14-DESIGN and SE-14-PROJECT.


\subsection{Objectives}
The Objective of this document is to provide a single way for JSON files and querys to be constructed
and interpreted.


\section{LAYOUT}

\subsection{Layout Information}
All values are stored as key, value pairs.

Values can be any of:
\begin{itemize}
\item Array containing values
\item Nested JSON object
\item string
\item number
\item true
\item false
\item null
\end{itemize}

All keys are strings and are required.

\subsection{Example}

\begin{verbatim}
	{
    	"data" 		: {
                        "walkname"			: "Name",
                    	"shortdescription"	: "Short Description",
                    	"longdescription"	: "Long Description",
                    	"time"				: 123455
                 	},
   		
        "points"	: {
        				"1" : {
                                "latitude"		: 52.122,
                                "longitude"		: -4.333,
                                "timestamp"		: 12345678,
                                "poiflag"		: true,
                                "poidata"		: {
                                                "locationname"	: "varsity",
                                                "description"	: "a good pub",
                                                "photo"			: [ "image_data",
                                                					"image_data2",
                                                                    "image_data3"
                                                                    ]
                                			}
                        	},
                     	"2" : {
                                "latitude"		: 52.122,
                                "longitude"		: -4.333,
                                "timestamp"		: 12345678,
                                "poiflag"		: false,
                                "poidata"		: null
                        	}
        			}
    
    }

\end{verbatim}

The first JSON Object contains the walk specific information such as the walk name, short description and long description and the timestamp of duration of the walk in seconds.

The second JSON Object contains contains JSON Objects representing the locations and points of interests.

Each location contains a longitude, latitude and a timestamp of the location. They also contain a flag
indicating whether it is a point of interest as well.

The poidata JSON Object contains the additional information that a point of interest requires such as the location name,
location description and an optional array of base64 image data strings.




%------------------- References
%--------- This contains  all the QA documents, edit out the ones you don't use in the document

\clearpage


%---------------------- Version History 
\addcontentsline{toc}{section}{DOCUMENT HISTORY}
\section*{DOCUMENT HISTORY}
\begin{flushleft}
\begin{tabular}{ | p{1.5cm} | p{1cm} | p{2cm} | p{6cm}| p{1.5cm}| }
\hline
Version & CCF No. & Date & Changes made to Document & Changed by \\
\hline

%----------- Add edits and change author
1.0 & N/A & 2014-1-28 & Initial creation & DAW46 \\
\hline

\end{tabular}
\end{flushleft}
\label{thelastpage}
\end{document}
